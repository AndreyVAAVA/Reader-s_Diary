\documentclass[a4paper,12pt]{article}
\usepackage{cmap}
\usepackage[T2A]{fontenc}
\usepackage[utf8]{inputenc}
\usepackage[english,russian]{babel}
\usepackage{amsmath,amsfonts,amsthm,mathtools} %AMS
\usepackage{icomma}%"умная запятая"
\usepackage{euscript} % Шрифт Евклид
\usepackage{mathrsfs} % Красивый матшрифт
\DeclareMathOperator{\sgn}{\mathop{sgn}}
\newcommand*{\hm}[1]{#1\nobreak\discretionary{}
	{\hbox{$\mathsurround=0pt #1$}}{}}
\author{Андрей Волков Александрович}
\title{Читательский дневник}
\date{26 августа 2020 г. - \today }

\begin{document} %конец преамбулы, начало текста.
	\maketitle
	Название книг в абзацах
	
	Перечисления идут в подабзацах(и не только перечисления)
	\section{Человек в футляре}
	\subsection{Главные герои: }
	Иван Иваныч, Буркин, Беликов, Михаил Коваленко, Варенька Коваленко 
	\subsection{Дата создания(и дополнительные подробности): }
	\maketitle
	<<Человек в футляре>> входит в серию <<Маленькая трилогия>>.
	
	\noindent
	Дата написания - 1898 г.
	
	\noindent
	Дата первой публикации - 1898 г.
	
	\subsection{Краткое описание сюжета: }
	\quad \, Рассказ начинается с описания ночлега двух охотников: Ивана Иваныча и Буркина. Они остановились в сарае старосты села и рассказывали друг другу разные истории. Разговор зашёл на тему людей «одиноких по натуре, которые, как рак-отшельник или улитка, стараются уйти в свою скорлупу». Буркин рассказывает историю о неком Беликове, недавно умершем в его городке. 
	
	Учитель Беликов преподает в гимназии древнегреческий язык, ведет уединенный образ жизни, и живет по принципу «Как бы чего не вышло».  Беликов вечно чем-то обеспокоен, и всегда выходит на улицу в неизменных очках, калошах и с зонтом. Огромное значение в его жизни имеет мнение окружающих людей, в особенности тех, кто стоит выше по должности. Его поведение не самым лучшим образом влияет на гимназию и всех жителей города. Беликов только тем и занимается, что внушает всем непреложную истину – жить нужно исключительно по правилам и циркулярам.

	Ситуация меняется, когда в гимназию устраивается новый учитель Коваленко. Беликов знакомится с его очаровательной сестрой Варенькой, и влюбляется в нее. Жизнерадостная, общительная, миловидная девушка вносит сумбур в размеренную жизнь героя. О чувствах Беликова становится известно всему городу, кто-то рисует карикатуру на влюбленного Беликова(<<Карикатура произвела на него самое тяжелое впечатление>>). Возмущению Беликова нет предала, когда он видит Вареньку на велосипеде, он уверен, что это верх неприличия. После Беликов приходит к дому, где жили Коваленко и хочет прочесть нравоучение, но Михаил Коваленко под смех Вареньки спускает его с лестницы.
	
	Мысль о пережитом унижении доводит Беликова до могилы. Лежа в гробу, герой выглядит счастливым – теперь он надежно защищен от мира крепким футляром.
	
	После всего в конце идёт рассуждение(между Буркиным и Иван Иванычем) о людях, подобных Беликову(о <<людях в футлярах>>)
	
	\section{Крыжовник}
	\subsection{Главные герои: }
	
\end{document} %Конец текста.