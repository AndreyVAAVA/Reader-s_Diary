\documentclass[a4paper,12pt]{article}
\usepackage{cmap}
\usepackage[T2A]{fontenc}
\usepackage[utf8]{inputenc}
\usepackage[english,russian]{babel}
\usepackage{amsmath,amsfonts,amsthm,mathtools} %AMS
\usepackage{icomma}%"умная запятая"
\usepackage{euscript} % Шрифт Евклид
\usepackage{mathrsfs} % Красивый матшрифт
\DeclareMathOperator{\sgn}{\mathop{sgn}}
\newcommand*{\hm}[1]{#1\nobreak\discretionary{}
	{\hbox{$\mathsurround=0pt #1$}}{}}
\author{Андрей Волков Александрович}
\title{Читательский дневник}
\date{26 августа 2020 г. - \today }

\begin{document} %конец преамбулы, начало текста.
	\maketitle
	Название книг в абзацах
	
	Перечисления идут в подабзацах(и не только перечисления)
	\section{"Человек в футляре"}
	\subsection{Главные герои: }
	Иван Иваныч, Буркин, Беликов, Михаил Коваленко, Варенька Коваленко 
	\subsection{Дата создания(и дополнительные подробности): }
	\maketitle
	<<Человек в футляре>> входит в серию <<Маленькая трилогия>>.
	
	\noindent
	Дата написания - 1898 г.
	
	\noindent
	Дата первой публикации - 1898 г.
	
	\subsection{Краткое описание сюжета: }
	\quad \, Рассказ начинается с описания ночлега двух охотников: Ивана Иваныча и Буркина. Они остановились в сарае старосты села и рассказывали друг другу разные истории. Разговор зашёл на тему людей «одиноких по натуре, которые, как рак-отшельник или улитка, стараются уйти в свою скорлупу». Буркин рассказывает историю о неком Беликове, недавно умершем в его городке. 
	
	Учитель Беликов преподает в гимназии древнегреческий язык, ведет уединенный образ жизни, и живет по принципу «Как бы чего не вышло».  Беликов вечно чем-то обеспокоен, и всегда выходит на улицу в неизменных очках, калошах и с зонтом. Огромное значение в его жизни имеет мнение окружающих людей, в особенности тех, кто стоит выше по должности. Его поведение не самым лучшим образом влияет на гимназию и всех жителей города. Беликов только тем и занимается, что внушает всем непреложную истину – жить нужно исключительно по правилам и циркулярам.

	Ситуация меняется, когда в гимназию устраивается новый учитель Коваленко. Беликов знакомится с его очаровательной сестрой Варенькой, и влюбляется в нее. Жизнерадостная, общительная, миловидная девушка вносит сумбур в размеренную жизнь героя. О чувствах Беликова становится известно всему городу, кто-то рисует карикатуру на влюбленного Беликова(<<Карикатура произвела на него самое тяжелое впечатление>>). Возмущению Беликова нет предала, когда он видит Вареньку на велосипеде, он уверен, что это верх неприличия. После Беликов приходит к дому, где жили Коваленко и хочет прочесть нравоучение, но Михаил Коваленко под смех Вареньки спускает его с лестницы.
	
	Мысль о пережитом унижении доводит Беликова до могилы. Лежа в гробу, герой выглядит счастливым – теперь он надежно защищен от мира крепким футляром.
	
	После всего в конце идёт рассуждение(между Буркиным и Иван Иванычем) о людях, подобных Беликову(о <<людях в футлярах>>)
	
	\section{"Крыжовник"}
	\subsection{Главные герои: }
	Иван Иванович, Николай Иванович.
	\subsection{Дата создания(и дополнительные подробности): }
	<<Крыжовник>> входит в серию <<Маленькая трилогия>>.
	
	\noindent
	Дата написания - 1898 г.
	
	\noindent
	Дата первой публикации - 1898 г.
	
	\subsection{Краткое описание сюжета: }
	\quad \, Разорившийся помещик Чимша-Гималайский умирает, и оставляет сыновьям родовое поместье, но скоро и его отбирают для погашения долгов.
	
	В отличии от брата, Николай мечтает о собственном доме. Символом уюта и благополучия для него становятся кусты крыжовника(<<Он чертил план своего имения, и всякий раз у него на плане выходило одно и то же: a) барский дом, b) людская, с) огород, d) крыжовник.>>). Николай много и упорно работает, откладывает деньги на покупку имения(тем самым отказывает себе во всём, даже в самых насущных потребностях.(<< Жил он скупо: недоедал, недопивал, одевался бог знает как, словно нищий, и всё копил и клал в банк. Страшно жадничал.>>))д
	
	Как только подворачивается возможность, Николай женится по расчету. Своей жесткой экономией он быстро доводит супругу до могилы. Вскоре Николай осуществляет свою мечту и приобретает усадьбу. (<<Николай Иваныч мало печалился; он выписал себе двадцать кустов крыжовника, посадил и зажил помещиком.>>)
	
	
	Спустя время Николая навещает Иван. Он замечает, что брат изменился в худшую сторону. Он заметно постарел, поправился, начал выражаться красивыми, но совершенно пустыми фразами. Николай прожил никчемную, серую жизнь, в которой не было никаких радостей и светлых моментов. Он не принес никому пользы, не совершил добрых поступков(точнее они были, но он их делал <<с важностью>>, и часто всё ограничивалось к примеру лечением содой, полвиною ведра алкоголя и пр.), хотя сам считал что <<Меня народ любит. Стоит мне только пальцем шевельнуть, и для меня народ сделает всё, что захочу.>>. Впрочем, Николай старается об этом особо не думать – он наслаждается своей усадьбой, и крыжовником.
	
	\section{"О любви"}
	\subsection{Главные герои: }
	Павел Константинович Алехин, Анна Алексеевна, Дмитрий Лугановичи.
	\subsection{Дата создания(и дополнительные подробности): }
	<<Крыжовник>> входит в серию <<Маленькая трилогия>>. (и завершает её)
	
	\noindent
	Дата написания - 1898 г.
	
	\noindent
	Дата первой публикации - 1898 г.
	
	\subsection{Краткое описание сюжета: }
	\quad \, После окончания учебы в университете, Алехин возвращается в отчий дом, но вместо наследства он получает лишь обязательства по долгам(<<когда я приехал сюда, был большой долг, а так как отец мой задолжал отчасти потому, что много тратил на мое образование, то я решил, что не уеду отсюда и буду работать, пока не уплачу этого долга.>>). Будучи деятельным человеком, Алехин начинает активно заниматься хозяйством (Он приводит дела в порядок, его начинают звать на суды(<<В первые же годы меня здесь выбрали в почетные мировые судьи>>)).
	
	Алехин все чаще выезжает в город, где после дела с поджигателями в суде знакомится с господином Лугановичем(<<В первые же годы меня здесь выбрали в почетные мировые судьи>>), который после приглашает его на обед и там же он знакомится с Анной Алексеевной(женой Лугановича)(<<Это было как раз после знаменитого дела поджигателей; разбирательство продолжалось два дня, мы были утомлены. Луганович посмотрел на меня и сказал:
	 — Знаете что? Пойдемте ко мне обедать.>>).
	Он видит в ней родственную душу, и его нежные чувства взаимны.(<<я видел женщину молодую, прекрасную, добрую, интеллигентную, обаятельную, женщину, какой я раньше никогда не встречал; и сразу я почувствовал в ней существо близкое, уже знакомое, точно это лицо, эти приветливые, умные глаза я видел уже когда-то в детстве>>)
	
	
	Следующая встреча Алехина и Лугановичей происходит спустя время на благотворительном балу. С тех пор он становится частым гостем в доме Лугановичей, его принимают за своего человека.(<<Мы беседовали подолгу и подолгу молчали, думая каждый о своем, или же она играла мне на рояле. Если же никого не было дома, то я оставался и ждал, разговаривал с няней, играл с ребенком или же в кабинете лежал на турецком диване и читал газету>>). Они все больше привязываются друг к другу, и вместе с тем страдают от своей любви.
	
	
	В течение долгих лет влюбленные не осмеливаются сделать решительный шаг. И лишь когда Лугановичи были вынуждены навсегда покинуть город, Алехин решает признаться в любви Анне Алексеевне. Их единственный поцелуй в купе поезда завершает так и неудавшийся роман.(<<Когда тут, в купе, взгляды наши встретились, душевные силы оставили нас обоих, я обнял ее, она прижалась лицом к моей груди, и слезы потекли из глаз; целуя ее лицо, плечи, руки, мокрые от слез, — о, как мы были с ней несчастны! — я признался ей в своей любви, и со жгучей болью в сердце я понял, как ненужно, мелко и как обманчиво было всё то, что нам мешало любить.>>)
	
	\section{"Ионыч"}
	\subsection{Главные герои: }
	Старцев Дмитрий Ионович, Туркин Иван Петрович, Туркина Вера Иосифовна, Туркина Екатерина Ивановна(Котик).
	\subsection{Дата создания(и дополнительные подробности): }
	В мае 1898 года, пройдя курс лечения на юге Франции, Чехов вернулся в Мелихово. Рассказ «Ионыч» был написан в паузах между строительными заботами, открытием новой школы и приёмом гостей.(и замсыел за время написания слегка изменился).
	\noindent
	Дата написания - 1898 г.
	
	\noindent
	Дата первой публикации - 1898 г.
	
	\subsection{Краткое описание сюжета: }
	\quad \, В губернском городе С. семья Туркиных считалась такой же достопримечательностью, как библиотека, театр или клуб. Глава семьи, Иван Петрович, устраивал любительские спектакли. Его жена, Вера Иосифовна, писала романы и повести. Дочь, Екатерина Ивановна, имевшая домашнее прозвище Котик, играла на рояле. Даже лакей Павлуша обладал актёрским талантом(<<Пава стал в позу, поднял вверх руку и проговорил трагическим тоном:
	
	 — Умри, несчастная!>>).
	
	Когда земский доктор Дмитрий Ионович Старцев поселился в Дялиже неподалёку от С., он был представлен Ивану Петровичу и приглашён в гости. Вечер, проведённый в доме Туркиных, прошёл душевно: пили чай, Вера Иосифовна читала вслух свой роман, начинавшийся словами «Мороз крепчал», Екатерина Ивановна музицировала. Дмитрий Ионович покинул Туркиных в хорошем расположении духа и без всякой усталости прошёл пешком девять вёрст до дому.
	
	Следующий визит Старцева в гостеприимный дом состоялся через много месяцев. Он заехал обследовать Веру Иосифовну, страдающую мигренями, и с тех пор стал наведываться к Туркиным при первой возможности(<<Вера Иосифовна давно уже страдала мигренью, но в последнее время, когда Котик каждый день пугала, что уедет в консерваторию, припадки стали повторяться всё чаще.>> <<Вера Иосифовна написала ему трогательное письмо, в котором просила его приехать и облегчить ее страдания. Старцев приехал и после этого стал бывать у Туркиных часто, очень часто... Он в самом деле немножко помог Вере Иосифовне, и она всем гостям уже говорила, что это необыкновенный, удивительный доктор. Но ездил он к Туркиным уже не ради ее мигрени...>>). Его по-настоящему увлекла Екатерина Ивановна; они подолгу беседовали о литературе и искусстве; неделя, проведённая без Котика, казалась Дмитрию Ионовичу вечностью. В один из дней девушка назначила ему свидание на кладбище. Старцев понимал, что это шутка, но всё равно в полночь приехал к памятнику Деметти и долго бродил в одиночестве между могил. На следующий день он сделал Екатерине Ивановне предложение и получил отказ: девушка объяснила, что жизнь в городе С. для неё невыносима, она хочет стать артисткой, посвятить себя искусству. Старцев переживал дня три, но потом как только узнал, что Котик уехала поступать в московскую консерваторию он успокоился.
	
	Через четыре года у Дмитрия Ионовича была уже большая практика; теперь он не ходил пешком, а ездил на тройке с бубенцами, потолстел. Однажды ему принесли письмо-приглашение от Туркиных; в их доме Старцев встретил Екатерину Ивановну. Она призналась, что великой пианистки из неё не получилось, зато Старцев в её глазах остаётся «лучшим из людей». Всё было как прежде: пили чай, Вера Иосифовна читала очередной роман. Этот визит к Туркиным оказался для Дмитрия Ионовича последним; больше они не встречались (хотя ему и приходили письма о от Екатерины, что нужно встретиться, но встречи так и не произошло, хоть Старцев и общеал).
	
	Спустя несколько лет у раздобревшего, погрусневшего доктора Старцева появились два дома и имение. Он стал легко раздражаться, в том числе на пациентов(<<Он одинок. Живется ему скучно, ничто его не интересует. За всё время, пока он живет в Дялиже, любовь к Котику была его единственной радостью и, вероятно, последней. По вечерам он играет в клубе в винт и потом сидит один за большим столом и ужинает.>>). В жизни Туркиных ничего не изменилось (<<Иван Петрович не постарел, нисколько не изменился и по-прежнему всё острит и рассказывает анекдоты; Вера Иосифовна читает гостям свои романы по-прежнему охотно, с сердечной простотой. А Котик играет на рояле каждый день, часа по четыре. Она заметно постарела, похварывает и каждую осень уезжает с матерью в Крым.>>). 
	
	\section{"Вишневый сад"}
	\subsection{Главные герои: }
	Любовь Андреевна Раневская, Лопахин Ермолай Алексеевич, Аня Раневская, Варя, Гаев Леонид Андреевич, Трофимов Петр Сергеевич
	\subsection{Дата создания(и дополнительные подробности): }
	«Вишнёвый сад» — последняя пьеса Чехова, завершённая на пороге первой русской революции, за год до его ранней смерти. Замысел пьесы возник у Чехова в начале 1901 года. Пьеса была закончена 26 сентября 1903 года.
	
	\noindent
	Дата написания - 1903 г.
	
	\noindent
	Дата первой публикации - 1904 г.
	
	\subsection{Краткое описание сюжета: }
	\quad \, Помещица Любовь Андреевна Раневская с дочерью Анной после долгой разлуки возвращаются домой. Все это время она жила в Париже, куда уехала после смерти супруга и маленького сына. За пять лет Любовь Андреевна тратит почти все свои сбережения. Тяжелым ударом для женщины становится предательство молодого любовника, который скрылся со всеми ее деньгами(в Париже).
	
	В родовом имении Любовь Андреевну встречает ее брат Леонид Андреевич Гаев и приемная дочь Варя. Они сообщают героине неприятную новость. Финансовое положение семьи настолько тяжелое, что необходимо продать имение и большой вишневый сад. Помещицу тут же встречает и потенциальный покупатель, богатый купец Лопахин, чьи родители были крепостными у помещиков Раневских. Любовь Андреевна наотрез отказывается продавать прекрасный вишневый сад, который был для нее символом счастливой, безмятежной юности.
	
	\noindent
	(<<Любовь Андреевна. О мой милый, мой нежный, прекрасный сад!.. Моя жизнь, моя молодость, счастье мое, прощай!.. Прощай!..
	
	Голос Ани (весело, призывающе): «Мама!..» Голос Трофимова (весело, возбужденно): «Ау!..»
	
	\noindent
	В последний раз взглянуть на стены, на окна... По этой комнате любила ходить покойная мать...>>)
	
	В день торгов Раневская устраивает в имении бал. Родовое поместье и вишневый сад уходят с молотка. Новым хозяином становится довольный Лопахин.(<<Я купил! Погодите, господа, сделайте милость, у меня в голове помутилось, говорить не могу... (Смеется.) Пришли мы на торги, там уже Дериганов. У Леонида Андреича было только пятнадцать тысяч, а Дериганов сверх долга сразу надавал тридцать. Вижу, дело такое, я схватился с ним, надавал сорок. Он сорок пять. Я пятьдесят пять. Он, значит, по пяти надбавляет, я по десяти... Ну, кончилось. Сверх долга я надавал девяносто, осталось за мной. Вишневый сад теперь мой! Мой! (Хохочет.) Боже мой, господи, вишневый сад мой! Скажите мне, что я пьян, не в своем уме, что все это мне представляется...>>) Анна находит общий язык с молодым учителем Трофимовым, который убеждает ее учиться и развиваться как личность(в течении всей пьесы). Любовь Андреевна возвращается в Париж, Гаев устраивается на работу в банк, а Варя получает должность экономки. Усадьба пустеет, приходит Фирс и вскоре в вишневом саду раздается звук топора.
	
	\section{"Чайка"}
	\subsection{Главные герои: }
	Любовь Андреевна Раневская, Лопахин Ермолай Алексеевич, Аня Раневская, Варя, Гаев Леонид Андреевич, Трофимов Петр Сергеевич
	\subsection{Дата создания(и дополнительные подробности): }
	«Чайка» — пьеса в четырёх действиях, впервые опубликованная в журнале «Русская мысль».
	
	\noindent
	Премьера состоялась 17 октября 1896 года на сцене петербургского Александринского театра. Стилизованный силуэт чеховской чайки по эскизу Фёдора Шехтеля стал эмблемой МХТ.
	
	\noindent
	Дата написания - 1895-1896 г.
	
	\noindent
	Дата первой публикации - 1896 г.
	
	\subsection{Краткое описание сюжета: }
	\quad \, Действие пьесы происходит в Российской Империи в конце XIX века. Отставной чиновник Сорин живет в своем имении. У него гостит его сестра Ирина Николаевна Аркадина и ее сын Константин Треплев. Госпожа Аркадина – 43-летняя известная актриса, ведущая бурную жизнь в обществе знаменитостей. Она состоит в отношениях с известным писателем Тригориным, который заметно моложе ее.
	
	Сын госпожи Аркадиной, Константин Треплев – 25-летний начинающий писатель. Он не одобряет образ жизни своей матери, ее вкусы и т.д. Сын с матерью состоят в непростых отношениях, хотя и любят друг друга. Треплев влюблен в соседку, молодую девушку Нину Заречную, и та отвечает ему взаимностью. Нина мечтает стать актрисой и хочет славы. Ее привлекает ее новый знакомый, писатель Тригорин, с которым она беседует о славе, успехе и прочие темы.
	
	Нина влюбляется в Тригорина, а тот увлекается ею. Константин ревнует и злится: ему кажется, что Нина разлюбила его из-за того, что он — писатель-неудачник. Тригорин признается госпоже Аркадиной, что влюблен в Нину. Влюбленная Аркадина на коленях упрашивает Тригорина не бросать. Бесхарактерный Тригорин остается с Аркадиной и вместе с ней собирается в Москву. Перед отъездом он тайно договаривается с Ниной о встрече в Москве.
	
	Вскоре Нина сбегает из дома в Москву и там вступает в любовную связь с Тригориным. У девушки рождается ребенок, который вскоре умирает. Нина становится актрисой, однако не добивается большого успеха(выступает в больших и важных представлениях, но ей не хватает таланта): Тригорин не поддерживает возлюбленную, из-за чего девушка теряет веру в себя. Тригорин изменяет Нине, а затем и вовсе охладевает к ней и бросает. Оставшись одна, несчастная Нина продолжает работать в мелких театрах и живет очень скромно.
	
	Так проходит два года. К этому времени Константин Треплев становится известным писателем. Он живет в том же имении вместе с дядей. Госпожа Аркадина возобнавляет отношения с Тригориным и, судя по всему, снова счастлива с ним. Однажды Аркадина и Тригорин приезжают в имение к Сорину. В усадьбе, как обычно, собираются знакомые.
	
	В этот же вечер Нина тайно навещает Конастантина (она уже 5 дней находится в городе). Девушка в слезах признается Треплеву, что боялась, что тот ее ненавидит, поэтому избегала встречи; завтра Нина едет в Елец, где всю зиму будет работать в местном театре. Треплев признается Нине, что все еще любит ее и что без нее его жизнь пуста.
	
	Он просит девушку остаться или взять его с собой. Нина отвечает, что по-прежнему любит Тригорина, хотя тот сломал ей жизнь и бросил ее. Девушка сравнивает себя с чайкой, которую случайный человек погубил от нечего делать. Нина и Константин прощаются. В тот же вечер, когда в доме еще находятся гости, Константин кончает жизнь самоубийством.(<<Направо за сценой выстрел; все вздрагивают.>> <<Дорн:... (Тоном ниже, вполголоса.) Уведите отсюда куда-нибудь Ирину Николаевну. Дело в том, что Константин Гаврилович застрелился...>>)
	
	
\end{document} %Конец текста.