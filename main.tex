\documentclass[a4paper,12pt]{article}
\usepackage{cmap}
\usepackage[T2A]{fontenc}
\usepackage[utf8]{inputenc}
\usepackage[english,russian]{babel}
\usepackage{amsmath,amsfonts,amsthm,mathtools} %AMS
\usepackage{icomma}%"умная запятая"
\usepackage{euscript} % Шрифт Евклид
\usepackage{mathrsfs} % Красивый матшрифт
\DeclareMathOperator{\sgn}{\mathop{sgn}}
\newcommand*{\hm}[1]{#1\nobreak\discretionary{}
	{\hbox{$\mathsurround=0pt #1$}}{}}
\author{Андрей Волков Александрович}
\title{Читательский дневник}
\date{26 августа 2020 г. - \today }

\begin{document} %конец преамбулы, начало текста.
	\maketitle
	Название книг в абзацах
	
	Перечисления идут в подабзацах(и не только перечисления)
	\section{"Человек в футляре"}
	\subsection{Главные герои: }
	Иван Иваныч, Буркин, Беликов, Михаил Коваленко, Варенька Коваленко 
	\subsection{Дата создания(и дополнительные подробности): }
	\maketitle
	<<Человек в футляре>> входит в серию <<Маленькая трилогия>>.
	
	\noindent
	Дата написания - 1898 г.
	
	\noindent
	Дата первой публикации - 1898 г.
	
	\subsection{Краткое описание сюжета: }
	\quad \, Рассказ начинается с описания ночлега двух охотников: Ивана Иваныча и Буркина. Они остановились в сарае старосты села и рассказывали друг другу разные истории. Разговор зашёл на тему людей «одиноких по натуре, которые, как рак-отшельник или улитка, стараются уйти в свою скорлупу». Буркин рассказывает историю о неком Беликове, недавно умершем в его городке. 
	
	Учитель Беликов преподает в гимназии древнегреческий язык, ведет уединенный образ жизни, и живет по принципу «Как бы чего не вышло».  Беликов вечно чем-то обеспокоен, и всегда выходит на улицу в неизменных очках, калошах и с зонтом. Огромное значение в его жизни имеет мнение окружающих людей, в особенности тех, кто стоит выше по должности. Его поведение не самым лучшим образом влияет на гимназию и всех жителей города. Беликов только тем и занимается, что внушает всем непреложную истину – жить нужно исключительно по правилам и циркулярам.

	Ситуация меняется, когда в гимназию устраивается новый учитель Коваленко. Беликов знакомится с его очаровательной сестрой Варенькой, и влюбляется в нее. Жизнерадостная, общительная, миловидная девушка вносит сумбур в размеренную жизнь героя. О чувствах Беликова становится известно всему городу, кто-то рисует карикатуру на влюбленного Беликова(<<Карикатура произвела на него самое тяжелое впечатление>>). Возмущению Беликова нет предала, когда он видит Вареньку на велосипеде, он уверен, что это верх неприличия. После Беликов приходит к дому, где жили Коваленко и хочет прочесть нравоучение, но Михаил Коваленко под смех Вареньки спускает его с лестницы.
	
	Мысль о пережитом унижении доводит Беликова до могилы. Лежа в гробу, герой выглядит счастливым – теперь он надежно защищен от мира крепким футляром.
	
	После всего в конце идёт рассуждение(между Буркиным и Иван Иванычем) о людях, подобных Беликову(о <<людях в футлярах>>)
	
	\section{"Крыжовник"}
	\subsection{Главные герои: }
	Иван Иванович, Николай Иванович.
	\subsection{Дата создания(и дополнительные подробности): }
	<<Крыжовник>> входит в серию <<Маленькая трилогия>>.
	
	\noindent
	Дата написания - 1898 г.
	
	\noindent
	Дата первой публикации - 1898 г.
	
	\subsection{Краткое описание сюжета: }
	\quad \, Разорившийся помещик Чимша-Гималайский умирает, и оставляет сыновьям родовое поместье, но скоро и его отбирают для погашения долгов.
	
	В отличии от брата, Николай мечтает о собственном доме. Символом уюта и благополучия для него становятся кусты крыжовника(<<Он чертил план своего имения, и всякий раз у него на плане выходило одно и то же: a) барский дом, b) людская, с) огород, d) крыжовник.>>). Николай много и упорно работает, откладывает деньги на покупку имения(тем самым отказывает себе во всём, даже в самых насущных потребностях.(<< Жил он скупо: недоедал, недопивал, одевался бог знает как, словно нищий, и всё копил и клал в банк. Страшно жадничал.>>))д
	
	Как только подворачивается возможность, Николай женится по расчету. Своей жесткой экономией он быстро доводит супругу до могилы. Вскоре Николай осуществляет свою мечту и приобретает усадьбу. (<<Николай Иваныч мало печалился; он выписал себе двадцать кустов крыжовника, посадил и зажил помещиком.>>)
	
	Спустя время Николая навещает Иван. Он замечает, что брат изменился в худшую сторону. Он заметно постарел, поправился, начал выражаться красивыми, но совершенно пустыми фразами. Николай прожил никчемную, серую жизнь, в которой не было никаких радостей и светлых моментов. Он не принес никому пользы, не совершил добрых поступков(точнее они были, но он их делал <<с важностью>>, и часто всё ограничивалось к примеру лечением содой, полвиною ведра алкоголя и пр.), хотя сам считал что <<Меня народ любит. Стоит мне только пальцем шевельнуть, и для меня народ сделает всё, что захочу.>>. Впрочем, Николай старается об этом особо не думать – он наслаждается своей усадьбой, и крыжовником.
	
	\section{"О любви"}
	\subsection{Главные герои: }
	Павел Константинович Алехин, Анна Алексеевна, Дмитрий Лугановичи.
	\subsection{Дата создания(и дополнительные подробности): }
	<<Крыжовник>> входит в серию <<Маленькая трилогия>>. (и завершает её)
	
	\noindent
	Дата написания - 1898 г.
	
	\noindent
	Дата первой публикации - 1898 г.
	
	\subsection{Краткое описание сюжета: }
	\quad \, После окончания учебы в университете, Алехин возвращается в отчий дом, но вместо наследства он получает лишь обязательства по долгам(<<когда я приехал сюда, был большой долг, а так как отец мой задолжал отчасти потому, что много тратил на мое образование, то я решил, что не уеду отсюда и буду работать, пока не уплачу этого долга.>>). Будучи деятельным человеком, Алехин начинает активно заниматься хозяйством (Он приводит дела в порядок, его начинают звать на суды(<<В первые же годы меня здесь выбрали в почетные мировые судьи>>)).
	
	Алехин все чаще выезжает в город, где после дела с поджигателями в суде знакомится с господином Лугановичем(<<В первые же годы меня здесь выбрали в почетные мировые судьи>>), который после приглашает его на обед и там же он знакомится с Анной Алексеевной(женой Лугановича)(<<Это было как раз после знаменитого дела поджигателей; разбирательство продолжалось два дня, мы были утомлены. Луганович посмотрел на меня и сказал:
	 — Знаете что? Пойдемте ко мне обедать.>>).
	Он видит в ней родственную душу, и его нежные чувства взаимны.(<<я видел женщину молодую, прекрасную, добрую, интеллигентную, обаятельную, женщину, какой я раньше никогда не встречал; и сразу я почувствовал в ней существо близкое, уже знакомое, точно это лицо, эти приветливые, умные глаза я видел уже когда-то в детстве>>)
	
	Следующая встреча Алехина и Лугановичей происходит спустя время на благотворительном балу. С тех пор он становится частым гостем в доме Лугановичей, его принимают за своего человека.(<<Мы беседовали подолгу и подолгу молчали, думая каждый о своем, или же она играла мне на рояле. Если же никого не было дома, то я оставался и ждал, разговаривал с няней, играл с ребенком или же в кабинете лежал на турецком диване и читал газету>>). Они все больше привязываются друг к другу, и вместе с тем страдают от своей любви.
	
	В течение долгих лет влюбленные не осмеливаются сделать решительный шаг. И лишь когда Лугановичи были вынуждены навсегда покинуть город, Алехин решает признаться в любви Анне Алексеевне. Их единственный поцелуй в купе поезда завершает так и неудавшийся роман.(<<Когда тут, в купе, взгляды наши встретились, душевные силы оставили нас обоих, я обнял ее, она прижалась лицом к моей груди, и слезы потекли из глаз; целуя ее лицо, плечи, руки, мокрые от слез, — о, как мы были с ней несчастны! — я признался ей в своей любви, и со жгучей болью в сердце я понял, как ненужно, мелко и как обманчиво было всё то, что нам мешало любить.>>)
	
	\section{"Ионыч"}
	\subsection{Главные герои: }
	Старцев Дмитрий Ионович, Туркин Иван Петрович, Туркина Вера Иосифовна, Туркина Екатерина Ивановна(Котик).
	\subsection{Дата создания(и дополнительные подробности): }
	В мае 1898 года, пройдя курс лечения на юге Франции, Чехов вернулся в Мелихово. Рассказ «Ионыч» был написан в паузах между строительными заботами, открытием новой школы и приёмом гостей.(и замсыел за время написания слегка изменился).
	\noindent
	Дата написания - 1898 г.
	
	\noindent
	Дата первой публикации - 1898 г.
	
	\subsection{Краткое описание сюжета: }
	\quad \, В губернском городе С. семья Туркиных считалась такой же достопримечательностью, как библиотека, театр или клуб. Глава семьи, Иван Петрович, устраивал любительские спектакли. Его жена, Вера Иосифовна, писала романы и повести. Дочь, Екатерина Ивановна, имевшая домашнее прозвище Котик, играла на рояле. Даже лакей Павлуша обладал актёрским талантом(<<Пава стал в позу, поднял вверх руку и проговорил трагическим тоном:
	
	 — Умри, несчастная!>>).
	
	Когда земский доктор Дмитрий Ионович Старцев поселился в Дялиже неподалёку от С., он был представлен Ивану Петровичу и приглашён в гости. Вечер, проведённый в доме Туркиных, прошёл душевно: пили чай, Вера Иосифовна читала вслух свой роман, начинавшийся словами «Мороз крепчал», Екатерина Ивановна музицировала. Дмитрий Ионович покинул Туркиных в хорошем расположении духа и без всякой усталости прошёл пешком девять вёрст до дому.
	
	Следующий визит Старцева в гостеприимный дом состоялся через много месяцев. Он заехал обследовать Веру Иосифовну, страдающую мигренями, и с тех пор стал наведываться к Туркиным при первой возможности(<<Вера Иосифовна давно уже страдала мигренью, но в последнее время, когда Котик каждый день пугала, что уедет в консерваторию, припадки стали повторяться всё чаще.>> <<Вера Иосифовна написала ему трогательное письмо, в котором просила его приехать и облегчить ее страдания. Старцев приехал и после этого стал бывать у Туркиных часто, очень часто... Он в самом деле немножко помог Вере Иосифовне, и она всем гостям уже говорила, что это необыкновенный, удивительный доктор. Но ездил он к Туркиным уже не ради ее мигрени...>>). Его по-настоящему увлекла Екатерина Ивановна; они подолгу беседовали о литературе и искусстве; неделя, проведённая без Котика, казалась Дмитрию Ионовичу вечностью. В один из дней девушка назначила ему свидание на кладбище. Старцев понимал, что это шутка, но всё равно в полночь приехал к памятнику Деметти и долго бродил в одиночестве между могил. На следующий день он сделал Екатерине Ивановне предложение и получил отказ: девушка объяснила, что жизнь в городе С. для неё невыносима, она хочет стать артисткой, посвятить себя искусству. Старцев переживал дня три, но потом как только узнал, что Котик уехала поступать в московскую консерваторию он успокоился.
	
	Через четыре года у Дмитрия Ионовича была уже большая практика; теперь он не ходил пешком, а ездил на тройке с бубенцами, потолстел. Однажды ему принесли письмо-приглашение от Туркиных; в их доме Старцев встретил Екатерину Ивановну. Она призналась, что великой пианистки из неё не получилось, зато Старцев в её глазах остаётся «лучшим из людей». Всё было как прежде: пили чай, Вера Иосифовна читала очередной роман. Этот визит к Туркиным оказался для Дмитрия Ионовича последним; больше они не встречались (хотя ему и приходили письма о от Екатерины, что нужно встретиться, но встречи так и не произошло, хоть Старцев и общеал).
	
	Спустя несколько лет у раздобревшего, погрусневшего доктора Старцева появились два дома и имение. Он стал легко раздражаться, в том числе на пациентов(<<Он одинок. Живется ему скучно, ничто его не интересует. За всё время, пока он живет в Дялиже, любовь к Котику была его единственной радостью и, вероятно, последней. По вечерам он играет в клубе в винт и потом сидит один за большим столом и ужинает.>>). В жизни Туркиных ничего не изменилось (<<Иван Петрович не постарел, нисколько не изменился и по-прежнему всё острит и рассказывает анекдоты; Вера Иосифовна читает гостям свои романы по-прежнему охотно, с сердечной простотой. А Котик играет на рояле каждый день, часа по четыре. Она заметно постарела, похварывает и каждую осень уезжает с матерью в Крым.>>). 
	
	\section{"Вишневый сад"}
	\subsection{Главные герои: }
	Любовь Андреевна Раневская, Лопахин Ермолай Алексеевич, Аня Раневская, Варя, Гаев Леонид Андреевич, Трофимов Петр Сергеевич
	\subsection{Дата создания(и дополнительные подробности): }
	«Вишнёвый сад» — последняя пьеса Чехова, завершённая на пороге первой русской революции, за год до его ранней смерти. Замысел пьесы возник у Чехова в начале 1901 года. Пьеса была закончена 26 сентября 1903 года.
	
	\noindent
	Дата написания - 1903 г.
	
	\noindent
	Дата первой публикации - 1904 г.
	
	\subsection{Краткое описание сюжета: }
	\quad \, Помещица Любовь Андреевна Раневская с дочерью Анной после долгой разлуки возвращаются домой. Все это время она жила в Париже, куда уехала после смерти супруга и маленького сына. За пять лет Любовь Андреевна тратит почти все свои сбережения. Тяжелым ударом для женщины становится предательство молодого любовника, который скрылся со всеми ее деньгами(в Париже).
	
	В родовом имении Любовь Андреевну встречает ее брат Леонид Андреевич Гаев и приемная дочь Варя. Они сообщают героине неприятную новость. Финансовое положение семьи настолько тяжелое, что необходимо продать имение и большой вишневый сад. Помещицу тут же встречает и потенциальный покупатель, богатый купец Лопахин, чьи родители были крепостными у помещиков Раневских. Любовь Андреевна наотрез отказывается продавать прекрасный вишневый сад, который был для нее символом счастливой, безмятежной юности.
	
	\noindent
	(<<Любовь Андреевна. О мой милый, мой нежный, прекрасный сад!.. Моя жизнь, моя молодость, счастье мое, прощай!.. Прощай!..
	
	Голос Ани (весело, призывающе): «Мама!..» Голос Трофимова (весело, возбужденно): «Ау!..»
	
	\noindent
	В последний раз взглянуть на стены, на окна... По этой комнате любила ходить покойная мать...>>)
	
	В день торгов Раневская устраивает в имении бал. Родовое поместье и вишневый сад уходят с молотка. Новым хозяином становится довольный Лопахин.(<<Я купил! Погодите, господа, сделайте милость, у меня в голове помутилось, говорить не могу... (Смеется.) Пришли мы на торги, там уже Дериганов. У Леонида Андреича было только пятнадцать тысяч, а Дериганов сверх долга сразу надавал тридцать. Вижу, дело такое, я схватился с ним, надавал сорок. Он сорок пять. Я пятьдесят пять. Он, значит, по пяти надбавляет, я по десяти... Ну, кончилось. Сверх долга я надавал девяносто, осталось за мной. Вишневый сад теперь мой! Мой! (Хохочет.) Боже мой, господи, вишневый сад мой! Скажите мне, что я пьян, не в своем уме, что все это мне представляется...>>) Анна находит общий язык с молодым учителем Трофимовым, который убеждает ее учиться и развиваться как личность(в течении всей пьесы). Любовь Андреевна возвращается в Париж, Гаев устраивается на работу в банк, а Варя получает должность экономки. Усадьба пустеет, приходит Фирс и вскоре в вишневом саду раздается звук топора.
	
	\section{"Чайка"}
	\subsection{Главные герои: }
	Любовь Андреевна Раневская, Лопахин Ермолай Алексеевич, Аня Раневская, Варя, Гаев Леонид Андреевич, Трофимов Петр Сергеевич
	\subsection{Дата создания(и дополнительные подробности): }
	«Чайка» — пьеса в четырёх действиях, впервые опубликованная в журнале «Русская мысль».
	
	\noindent
	Премьера состоялась 17 октября 1896 года на сцене петербургского Александринского театра. Стилизованный силуэт чеховской чайки по эскизу Фёдора Шехтеля стал эмблемой МХТ.
	
	\noindent
	Дата написания - 1895-1896 г.
	
	\noindent
	Дата первой публикации - 1896 г.
	
	\subsection{Краткое описание сюжета: }
	\quad \, Действие пьесы происходит в Российской Империи в конце XIX века. Отставной чиновник Сорин живет в своем имении. У него гостит его сестра Ирина Николаевна Аркадина и ее сын Константин Треплев. Госпожа Аркадина – 43-летняя известная актриса, ведущая бурную жизнь в обществе знаменитостей. Она состоит в отношениях с известным писателем Тригориным, который заметно моложе ее.
	
	Сын госпожи Аркадиной, Константин Треплев – 25-летний начинающий писатель. Он не одобряет образ жизни своей матери, ее вкусы и т.д. Сын с матерью состоят в непростых отношениях, хотя и любят друг друга. Треплев влюблен в соседку, молодую девушку Нину Заречную, и та отвечает ему взаимностью. Нина мечтает стать актрисой и хочет славы. Ее привлекает ее новый знакомый, писатель Тригорин, с которым она беседует о славе, успехе и прочие темы.
	
	Нина влюбляется в Тригорина, а тот увлекается ею. Константин ревнует и злится: ему кажется, что Нина разлюбила его из-за того, что он — писатель-неудачник. Тригорин признается госпоже Аркадиной, что влюблен в Нину. Влюбленная Аркадина на коленях упрашивает Тригорина не бросать. Бесхарактерный Тригорин остается с Аркадиной и вместе с ней собирается в Москву. Перед отъездом он тайно договаривается с Ниной о встрече в Москве.
	
	Вскоре Нина сбегает из дома в Москву и там вступает в любовную связь с Тригориным. У девушки рождается ребенок, который вскоре умирает. Нина становится актрисой, однако не добивается большого успеха(выступает в больших и важных представлениях, но ей не хватает таланта): Тригорин не поддерживает возлюбленную, из-за чего девушка теряет веру в себя. Тригорин изменяет Нине, а затем и вовсе охладевает к ней и бросает. Оставшись одна, несчастная Нина продолжает работать в мелких театрах и живет очень скромно.
	
	Так проходит два года. К этому времени Константин Треплев становится известным писателем. Он живет в том же имении вместе с дядей. Госпожа Аркадина возобнавляет отношения с Тригориным и, судя по всему, снова счастлива с ним. Однажды Аркадина и Тригорин приезжают в имение к Сорину. В усадьбе, как обычно, собираются знакомые.
	
	В этот же вечер Нина тайно навещает Конастантина (она уже 5 дней находится в городе). Девушка в слезах признается Треплеву, что боялась, что тот ее ненавидит, поэтому избегала встречи; завтра Нина едет в Елец, где всю зиму будет работать в местном театре. Треплев признается Нине, что все еще любит ее и что без нее его жизнь пуста.
	
	Он просит девушку остаться или взять его с собой. Нина отвечает, что по-прежнему любит Тригорина, хотя тот сломал ей жизнь и бросил ее. Девушка сравнивает себя с чайкой, которую случайный человек погубил от нечего делать. Нина и Константин прощаются. В тот же вечер, когда в доме еще находятся гости, Константин кончает жизнь самоубийством.(<<Направо за сценой выстрел; все вздрагивают.>> <<Дорн:... (Тоном ниже, вполголоса.) Уведите отсюда куда-нибудь Ирину Николаевну. Дело в том, что Константин Гаврилович застрелился...>>)
	
	\section{"Анна Снегина"}
	\subsection{Главные герои: }
	Сергуша, Анна Снегина
	\subsection{Дата создания(и дополнительные подробности): }
	«Анна Снегина» — автобиографическая поэма Сергея Есенина. В основу легли воспоминания поэта о том, как он посетил родное село, о революции, о безответной любви в юности. Поэма посвящена А. Воронскому. Впервые отрывки из поэмы были опубликованы весной 1925 года в журнале «Город и деревня». Полностью поэму напечатали в газете «Бакинский рабочий» в № 95 и № 96 первого и третьего мая.
	
	\noindent
	Сам Есенин определил жанр "на глазок": лироэпическая поэма. Некоторые исследователи считают, что это определение не совсем точно выражает её жанровое своеобразие. В.Турбин, к примеру, называет «Анну Снегину» «повестью в стихах» и находит в ней сходство с «Евгением Онегиным». По мысли Турбина, на сходство намекает и соотнесенность, внутренняя зарифмованность названий: О-негин, С-негина. Ещё одно определение предложил А. Квятковский, автор авторитетного «Поэтического словаря»: последняя крупная вещь Есенина — стихотворная новелла, то есть повествование с напряжённым романным сюжетом и неожиданной концовкой. 
	
	\noindent
	Дата написания - 1924-1925 г.
	
	\noindent
	Дата первой публикации - 1925 г.
	
	\subsection{Краткое описание сюжета: }
	\quad \, Молодой поэт (подразумевается сам автор) возвращается в родное село Радово (сам же поэт приезжал в родное Константиново летом 1917—1918 годов), устав от бурных революционных событий. Герой отмечает, какие изменения произошли в его селе. Вскоре он встречается с крестьянами из Криуши, в частности с Проном Оглоблиным. Те расспрашивают знаменитого поэта, приехавшего из столицы, о положении дел, о том, кто же такой Ленин. Позже к поэту приезжает молодая помещица Анна Снегина, в которую он был влюблён в шестнадцатилетнем возрасте. Они вспоминают прошлое, молодость, свои мечты. Спустя какое-то время, Сергуша приезжает в Криушу по просьбе Прона и оказывается вовлечённым в бунт: криушские крестьяне требуют от Снегиной отдать землю. Приходит известие о том, что на войне убили Борю — мужа Снегиной. Снегина в гневе на поэта, которому доверяла. Земля достаётся крестьянам. Проходит время. Анна, попросив прощение за обиду, уезжает вместе с матерью. Лирический герой возвращается в Петербург. А через некоторое время получает письмо от мельника, в котором говорится, что Прон Оглоблин был расстрелян отрядом Деникина. Приехав навестить мельника, поэт получает второе письмо, от Анны из Лондона.
	
	\noindent
	(<<«Вы живы?.. Я очень рада...
	
	\noindent
	Я тоже, как вы, жива.
	
	\noindent
	Так часто мне снится ограда,
	
	\noindent
	Калитка и ваши слова.
	
	\noindent
	Теперь я от вас далеко...
	
	\noindent
	В России теперь апрель.
	
	\noindent
	И синею заволокой
	
	\noindent
	Покрыта береза и ель.
	
	\noindent
	Сейчас вот, когда бумаге
	
	\noindent
	Вверяю я грусть моих слов,
	
	\noindent
	Вы с мельником, может, на тяге
	
	\noindent
	Подслушиваете тетеревов,
	
	\noindent
	Я часто хожу на пристань
	
	\noindent
	И, то ли на радость, то ль в страх,
	
	\noindent
	Гляжу средь судов все пристальней
	
	\noindent
	На красный советский флаг.
	
	\noindent
	Теперь там достигли силы.
	
	\noindent
	Дорога моя ясна...
	
	\noindent
	Но вы мне по-прежнему милы,
	
	\noindent
	Как родина и как весна».>>)
	
	\section{"Двенадцать"}
	\subsection{Главные герои: }
	Отряд из двенадцати(красногвардейцы), Ванька, Катька, Петруха
	\subsection{Дата создания(и дополнительные подробности): }
	«Двенадцать» — поэма Александра Блока, одна из признанных вершин его творчества и русской поэзии в целом. 
	
	\noindent
	Сразу же после публикации и первых концертов поэма была принята в штыки большинством представителей русской интеллигенции. Многие из бывших поклонников и даже друзей Блока порвали с ним всякие отношения, что объясняется напряжённой атмосферой (особенно в первые зимние месяцы) после Октябрьской Революции.
	
	\noindent
	В апреле 1920 года Блок добавляет эти слова, полные внутренней борьбы и сомнения: «Оттого я и не отрекаюсь от написанного тогда, что оно было написано в согласии со стихией…» Тем не менее, через год, в предсмертном бреду Блок требовал от своей жены обещания сжечь и уничтожить все до единого экземпляры поэмы «Двенадцать». Это было напрямую связано с той эволюцией в отношении Блока к революции и большевикам, которую он прошёл после создания поэмы.
	
	\noindent
	Дата написания - 1918 г.
	
	\noindent
	Дата первой публикации - 1918 г.
	
	\subsection{Краткое описание сюжета: }
	\quad \, Первая глава представляет собой типическую экспозицию (завязку) сюжета — заснеженные улицы революционного Петрограда зимой 1917—1918. Кратко и ёмко набросаны портреты нескольких прохожих — вот священник, вот богатая женщина в каракуле, старухи… По улицам замёрзшего города идёт патрульный отряд революционеров из двенадцати человек. Патрульные обсуждают своего бывшего товарища Ваньку, бросившего дело революции(он ушел и теперь стал солдатом
	
	<<— Ванюшка сам теперь богат...
	
	\, — Был Ванька наш, а стал солдат!>>
	
	\noindent
	), после вернувшегося ради кабаков и Катьки, а также поют песню о службе в Красной гвардии. Неожиданно отряд сталкивается с повозкой, на которой едут Ванька с Катькой. Красногвардейцы нападают на сани; извозчику и Ваньке удаётся выскочить из-под огня(и сбежать), но Катька погибает от выстрела одного из двенадцати(
	
	<<А Катька где? — Мертва, мертва!
	
	\quad  \quad \quad Простреленная голова!>>
	
	\noindent
	). Убивший её боец Петруха печалится, но товарищи осуждают его за это. Патруль идёт дальше, держа революционный шаг. За ними увязывается шелудивый пёс, но его отгоняют штыками. Затем бойцы видят впереди неясную фигуру — <<Впереди - Иисус Христос>>.
	
	Говоря словами Блока, «Двенадцать» сосредоточили в себе всю «электрическую» силу, которой был перенасыщен воздух Октября. (<<Товарищ, винтовку держи, не трусь! Пальнём-ка пулей в Святую Русь…>>).
	
	\section{"Во весь голос"}
	\subsection{Главные герои: }
	Владимир Владимирович Маяковский
	\subsection{Дата создания(и дополнительные подробности): }
	»Во весь голос» — последняя поэма Владимира Маяковского, первое вступление которой было написано в декабре 1929 — январе 1930 года.
	
	\noindent
	Во время работы над поэмой «Во весь голос» Маяковский занимался подготовкой своей юбилейной выставки «20 лет работы». Об этом свидетельствует его выступление 25 марта 1930 года в Доме комсомола Красной Пресни, посвящённом двадцатилетию его деятельности, где он подробно изложил замысел своей новой поэмы
	
	
	\noindent
	Поэма имеет подзаголовок «Первое вступление в поэму», ибо задумывалась как «вступление» к будущей поэме о пятилетке. Тем не менее, «Во весь голос» является законченным произведением и Маяковский рекомендовал её как «свою поэму». «Во весь голос» был написан Маяковским в период жёсткой критики его деятельности, которая стала поводом для создания произведения с «правильным взглядом» на его творчество. По мнению некоторых исследователей, поэма посвящена теме борьбы с РАППом или конструктивистами
	
	\noindent
	Дата написания - 1929-1930 г.
	
	\noindent
	Дата первой публикации - 1930 г.(Но это лишь первое вступление(и даже оно не полностью написано), к сожалению из-за самоубийства писателя мы ничего дальше этого не имеем)
	
	\subsection{Краткое описание сюжета: }
	\quad \, В творении «Во весь голос» присутствуют писательская страсть, глубокие мысли о времени, жизни самого поэта. Также автор дает понять значение революций и искусства в его жизни. Книга описывает его участия в революциях за свободу партии и государства. Поэзию писатель применял как оружие против государства. Эта книга – конец творческой жизни и пути поэта. На своем пути автор прошел сложный путь. В последней поэме поэт показывает силу слов, которые провоцируют людей к действию. С этими словами его литературные противники не могут понять, что поэзия – это дело всей жизни Маяковского.
	
	Вступительную речь поэмы «Во весь голос» писатель так и не смог закончить. Во вступлении слова и голос автора являются центральным образом. Другие упрекнули автора в эгоцентризме. Поскольку автор не может воспринимать себя как центральный образ, вокруг которого вращается вселенная. А автор считал себя призванным человеком к революции. В этом творении автор скрыл полемику с поэзией Есенина. Есенин описывал необычайный пейзаж и любовную лирику.
	
	\section{"Облако в штанах"}
	\subsection{Главные герои: }
	Двадцатидвухлетний поэт, Мария.
	\subsection{Дата создания(и дополнительные подробности): }
	
	Изначально она называлась «Тринадцатый апостол», но, по требованию царской цензуры, название было изменено поэтом на «Облако в штанах»
	
	\noindent
	В предисловии к первому полному изданию поэмы в 1918 году Владимир Маяковский написал: «„Облако в штанах“ (первое имя „Тринадцатый апостол“ зачеркнуто цензурой. Не восстанавливаю. Свыкся) считаю катехизисом сегодняшнего искусства; „Долой вашу любовь“, „долой ваше искусство“, „долой ваш строй“, „долой вашу религию“ — четыре крика четырёх частей». Поэма имеет маленькое вступление. 
	
	\noindent
	Дата написания - 1914-1915 г.
	
	\noindent
	Дата первой публикации - 1915 г.
	
	\subsection{Краткое описание сюжета: }
	\quad \, Молодой человек ждет, стоя у окна, девушку, которую он любит. Он отмеряет часы, нервы его раскалены до предела. Она должна была появиться часам к четырем, но уже вечер, а ее все нет. Проходит еще не один час, прежде он услышал ее шаги. Сердце юноши вырывается из груди. И вот она приходит, сжимая перчатки в руках, она говорит герою о том, что выходит замуж. Юноша подавлен, но стойко выносит плохую весть. Он не показывает, как ему больно. Его захлестнула неимоверная боль, которую он не смел показать любимой девушке. Герой страдает и винит во всем случившемся Бога. Он понимает, что его сердце полностью принадлежит ей, что он безумно ее любит. И ревность завладевает им.
	
	Автор описывает, что с ним происходит, прибегаю ко множеству слов, чтобы нам стало немного яснее, что же он на самом деле испытывает. Он понимает, что он безумно влюблен и не сможет полюбить другую.
	
	Автор думает, будет ли в его жизни еще любовь. Еще его посещает мысль: если будет любовь, то какая? Может искренняя, всепоглощающая или же неяркая, маленькая. Герой думает о том, что любовь оставила ему только разбитое сердце. Он говорит, что настоящая любовь способна приносить не только радость и счастье, но и настоящую, жгучую боль.
	
	Автор стихотворения считает, что без любви невозможна жизнь. Что любовь – это и есть сама жизнь. Он говорит зло о любви, показывая сколько страдания она принесла его сердцу, но все же во всем произведении мы видим, что все же была радость у него от этого чувства. Что когда-то он любил и был счастлив.
	
	Автор говорит, что его душа спрятана от глаз чужих. Ведь нельзя ходить с обнаженной душой, потому что там таится все самое сокровенное. Там покоятся настоящие чувства, которые он прячет под гримасой безразличия. Так никто не сможет понять, что же на самом деле творится у поэта на сердце.
	
	Также Маяковский обращается к поэзии. Он говорит о том, что она перестала быть отражением душ людей, что она стала продуманной, вычищенной от «лишних» слов. Говорит о том, что поэзия стала мертвой.
	
	Герой надеется на благосклонность небес, но вокруг тишина.(
	
	<<Эй, вы!
	
	Небо!
	
	Снимите шляпу!
	
	Я иду!
	
	Глухо.
	
	Вселенная спит,
	
	положив на лапу
	
	с клещами звезд огромное ухо.>>)
	
	\section{"Иуда Искариот"}
	\subsection{Главные герои: }
	Иуда из Кариота, Иисус Христос (Назорей), Апостолы.
	\subsection{Дата создания(и дополнительные подробности): }
	«Иуда Искариот» — повесть русского писателя-экспрессиониста Леонида Андреева, впервые опубликованная под заглавием «Иуда Искариот и другие» в альманахе «Сборник товарищества „Знание“ за 1907 год», книга 16. 
	
	\noindent
	Дата написания - 1906-1907 г.
	
	\noindent
	Дата первой публикации - 1907 г.
	
	\subsection{Краткое описание сюжета: }
	\quad \, К Иисусу и странствующим с ним апостолам прибивается человек по имени Иуда из города Кариота. Об этом бродяге идет дурная слава, что он вор, мошенник, преступник. Что бедная жена его, брошенная в нищете, вынуждена скитаться по домам выпрашивая подаяний. Иуда безобразен с виду, он отталкивает от себя всех, но Иисус обращается с ним ласково, чем вызывает в апостолах еще больший гнев.
	
	Язык у Иуды без костей, а сердце без жалости. Всех он презирает, не верит в честность и добродетель. По словам Иуды, каждый человек в чем-то да повинен, а тот, кто утверждает иначе - лжет. Своим отцом он называет козла или самого Дьявола, однако к Иисусу, Иуда относится с большой нежностью. Ближе всех они сходятся с Фомой, тот заинтересован поведением и образом мыслей Иуды.
	
	Чтобы показать свое доверие к Иуде, Иисус поручает ему отвечать за деньги, которые подают апостолам. Иуда поносит тех, кто приходит глазеть на Иисуса, утверждая, что они ничего не поймут и переврут его слова. Однажды Фома тайком наведывается в покинутый ими поселок и убеждается, что так оно и есть. В другой раз иуда хитростью спасает Иисуса и апостолов от гнева толпы.
	
	Как-то на отдыхе апостолы бросают тяжелые камни, похваляясь силой. Петр получает звание самого могучего, но Иуда побеждает его.
	
	Фома уличает Иуду в краже общественных денег. Тот признается, что отдал их продажной девице за две ночи, проведенные с нею. Иисус говорит, что Иуда может брать столько, сколько ему потребно.
	
	Иоанн и Петр спорят о том, кому из них быть одесную с Иисусом в Царствии Небесном. Иуда берется их рассудить, каждому суля первенство, но уверен, что рядом с Христом будет он сам.
	
	Задумав предать Иисуса на смерть он убеждает первосвященника Анну поймать и судить его. Тот сначала отказывается, но потом сулит Иуде за предательство сумму в тридцать серебряных монет. Иуда возмущен таким ничтожным гонораром, однако в конце концов соглашается.
	
	Совершив предательство, Иуда изо всех сил старается угодить Иисусу, сделать радостным отпущенное ему время. Он сам убеждает других апостолов, что они должны беречь учителя, приносит им оружие.
	
	
\end{document} %Конец текста.